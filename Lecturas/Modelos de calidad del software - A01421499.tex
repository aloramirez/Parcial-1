% Generated by GrindEQ Word-to-LaTeX 
\documentclass{article} %%% use \documentstyle for old LaTeX compilers

\usepackage[english]{babel} %%% 'french', 'german', 'spanish', 'danish', etc.
\usepackage{amssymb}
\usepackage{amsmath}
\usepackage{txfonts}
\usepackage{mathdots}
\usepackage[classicReIm]{kpfonts}
\usepackage[dvips]{graphicx} %%% use 'pdftex' instead of 'dvips' for PDF output

% You can include more LaTeX packages here 


\begin{document}

%\selectlanguage{english} %%% remove comment delimiter ('%') and select language if required


\noindent \textbf{Modelos de calidad del software, un estado del arte}

\noindent \textbf{Calidad de software}

\begin{enumerate}
\item \textbf{ }Esto habla del grado de desempe\~{n}o de las principales caracter\'{i}sticas con las que debe cumplir un sistema durante su ciclo de vida. Por ello es importante implementar un modelo o est\'{a}ndar de calidad que permita gestionar los tributos en el proceso de construcci\'{o}n del software (dicho proceso de construcci\'{o}n son las medidas de calidad establecidas)
\end{enumerate}

\noindent \textbf{Modelos de calidad de software}

\begin{enumerate}
\item \textbf{ }Estos modelos deben enfocarse en hacer seguimiento y evaluaci\'{o}n cada etapa de construcci\'{o}n del producto. Es decir que la organizaci\'{o}n lleve una documentaci\'{o}n del proceso llevado a cabo.
\end{enumerate}

\noindent \textbf{Estructura y enfoque de los modelos de calidad de software}

\begin{enumerate}
\item \textbf{ }Estos se clasifican dependiendo el enfoque de evaluaci\'{o}n, ya sea a nivel de procesos, producto o calidad en uso.
\end{enumerate}

\noindent \textbf{Calidad a nivel de proceso}

\begin{enumerate}
\item \textbf{ }La calidad de un sistema debe ser discutida desde el principio y posteriormente en cada etapa del proceso se debe llevar a cabo el control y seguimiento de los aspectos de calidad para minimizar los riesgos y poder ofrecer un soporte continuo.
\end{enumerate}

\noindent \textbf{Calidad nivel producto }

\begin{enumerate}
\item \textbf{ }Este tipo de calidad se trata de especificar y evaluar el cumplimiento de criterios del producto ya sea con medidas internas o externas. Por lo cual se han definido tres tipos de medidas de calidad: internas, externas y en uso.
\end{enumerate}

\noindent \textbf{Calidad en uso}

\begin{enumerate}
\item \textbf{ }Es el conjunto de atributos relacionados con la aceptaci\'{o}n del usuario final y seguridad.
\end{enumerate}

\noindent \textbf{Modelos a nivel de proceso}

\begin{enumerate}
\item \textbf{ }ITIL, ISO / IEC 15504, \textbf{Bootstrap}, Dromey, PSP, TSP, \textbf{IEEE / EIA 12207}, Cobit 4.0, ISO 90003, \textbf{CMMI} e ISO / IEC 20000.
\end{enumerate}

\noindent \textbf{Modelos a nivel producto}

\begin{enumerate}
\item \textbf{ }McCall, GQM, Boehm, FURPS, GILB, ISO 9126, SQAE, WebQEM e ISO 25000.
\end{enumerate}

\noindent Experiencias de implementaci\'{o}n de modelos de calidad de software.

\begin{enumerate}
\item  El CMMI es el que ha tenido mejor respuesta por parte de las grandes empresas. Por otra parte, Bootstrap se ha implementado en empresas europeas. Mientras que PSP ha sido usado m\'{a}s en \'{a}mbitos acad\'{e}micos, desarrollo de software y mejora de procesos empresariales. TSP esta enfocada a soluciones de negocios. Por otro lado, ISO 90003 fue usado para reducir defectos e incidencias, aumento de productividad, compromiso con el cliente y una mejora continua en los servicios.  Esto solo por mencionar algunos y los usos que se les han dado.
\end{enumerate}

\noindent Algunos de estos modelos han sido clave para modelos recientes y gracias a ellos, los nuevos modelos est\'{a}n mas completos que nunca. Por ello es importante que las empresas dedicadas al software se certifiquen para demostrar su buen desempe\~{n}o en el \'{a}rea. 

\noindent 


\end{document}

