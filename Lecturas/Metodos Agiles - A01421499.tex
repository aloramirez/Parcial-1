% Generated by GrindEQ Word-to-LaTeX 
\documentclass{article} %%% use \documentstyle for old LaTeX compilers

\usepackage[english]{babel} %%% 'french', 'german', 'spanish', 'danish', etc.
\usepackage{amssymb}
\usepackage{amsmath}
\usepackage{txfonts}
\usepackage{mathdots}
\usepackage[classicReIm]{kpfonts}
\usepackage[dvips]{graphicx} %%% use 'pdftex' instead of 'dvips' for PDF output

% You can include more LaTeX packages here 


\begin{document}

%\selectlanguage{english} %%% remove comment delimiter ('%') and select language if required


\noindent \textbf{Metodolog\'{i}as \'{A}giles de Desarrollo de Software Aplicadas a la Gesti\'{o}n de Proyectos Empresariales}

\begin{enumerate}
\item \textbf{ }Hoy en d\'{i}a ha nacido un creciente inter\'{e}s por las metodolog\'{i}as de desarrollo de software que agilicen el tiempo de desarrollo y garanticen el uso eficiente de los recursos, aplicadas tanto para empresas grandes con numerosos procesos como a empresas peque\~{n}as que no cuentan con muchas herramientas para llevar a cabo los proyectos.

\item  Las metodolog\'{i}as agiles surgen de una iniciativa con la participaci\'{o}n de un grupo de 17 expertos en el \'{a}rea de desarrollo de software, los cuales manifestaron la importancia que el equipo desarrollador respondiera de forma oportuna a los cambios que puedan surgir a lo largo de la ejecuci\'{o}n del proyecto.

\item  Las organizaciones que quieran generar efectos diferenciadores con respecto a sus competidores deben implementar software en el desarrollo de sus actividades de negocio, todo esto a trav\'{e}s de la gesti\'{o}n de proyectos.

\item  Un Proyecto Inform\'{a}tico es un sistema de cursos de acciones simult\'{a}neas y/o secuenciales que incluye personas, equipamientos de hardware, software y comunicaciones, enfocadas en obtener uno o m\'{a}s resultados deseables sobre un sistema de informaci\'{o}n.
\end{enumerate}

\noindent \textbf{Las Fases principales de un Proyecto de desarrollo de Software}

\begin{enumerate}
\item \textbf{ }Planeaci\'{o}n

\begin{enumerate}
\item  Definici\'{o}n del problema

\item  Planificaci\'{o}n del proyecto

\begin{enumerate}
\item  Calidad

\item  Costo

\item  Tiempo de duraci\'{o}n
\end{enumerate}
\end{enumerate}

\item  Ejecuci\'{o}n

\begin{enumerate}
\item  Puesta en marcha

\item  Fase productiva

\item  Conclusi\'{o}n del proyecto
\end{enumerate}

\item  Soporte

\begin{enumerate}
\item  Mantenimiento del software
\end{enumerate}
\end{enumerate}

\noindent \textbf{Manifest\'{o} \'{a}gil}

\noindent Es un documento que habla de los principios y valores que hacen diferente a un proyecto de desarrollo tradicional y a uno con un modelo \'{a}gil. Que adem\'{a}s ayuda a que se puedan solucionar los problemas que surjan en tiempo y forma.

\noindent \textbf{Principales metodolog\'{i}as agiles }

\begin{enumerate}
\item \textbf{ }\textit{Scrum}

\begin{enumerate}
\item \textit{ }Scrum se basa en la teor\'{i}a de control de procesos emp\'{i}ricos. Esta metodolog\'{i}a tiene un enfoque iterativo e incremental para optimizar la predicci\'{o}n de futuros problemas y evitar riesgos mayores.  Los eventos importantes de la metodolog\'{i}a son:

\begin{enumerate}
\item  Reuni\'{o}n de planificaci\'{o}n del sprint

\item  Scrum Diario

\item  Revisi\'{o}n del Sprint

\item  Retrospectiva del sprint
\end{enumerate}

\item  Esta metodolog\'{i}a tiene diferentes papeles como: Scrum Master, Product Owner, Team Members, Users y Skateholders.
\end{enumerate}

\item  \textit{Extreme Programming}

\begin{enumerate}
\item \textit{ } Esta metodolog\'{i}a se basa en una serie de reglas y principios, de manera en la que se les agreguen valor y quiten procedimientos que generan burocracia en el mismo.

\item  Esta metodolog\'{i}a se engloba en 12 principios que a su vez se agrupan en 4 categor\'{i}as:

\begin{enumerate}
\item  Retroalimentaci\'{o}n

\item  Proceso Continuo en lugar de ir por lotes

\item  Entendimiento Compartido

\item  Bienestar del programador
\end{enumerate}
\end{enumerate}

\item  \textit{Crystal Clear}

\begin{enumerate}
\item \textit{ }Esta metodolog\'{i}a establece c\'{o}digos de color como parte de la definici\'{o}n de la complejidad de la misma, si es m\'{a}s obscuro, el m\'{e}todo suele ser m\'{a}s pesado. 

\item  Esta metodolog\'{i}a se rige por principios que hacen correcto su funcionamiento.

\begin{enumerate}
\item  Cada proyecto necesita un grado diferente de compensaci\'{o}n.

\item  Entre m\'{a}s peque\~{n}o el proyecto, mejor coordinados

\item  Cada proyecto tiene su propio medio de comunicaci\'{o}n

\item  Debe haber retroalimentaci\'{o}n y comunicaci\'{o}n efectiva.
\end{enumerate}

\item  En esta metodolog\'{i}a hay 8 roles asignados a los miembros del equipo, para su correcto desempe\~{n}o:

\begin{enumerate}
\item  Sponsor

\item  Equipo

\item  Coordinador

\item  Experto del negocio

\item  Lider de dise\~{n}o

\item  Dise\~{n}adores y programadores

\item  Tester
\end{enumerate}
\end{enumerate}
\end{enumerate}

\noindent Plataformas y arquitecturas

\begin{enumerate}
\item  Debido a las diversas metodolog\'{i}as agiles, tambi\'{e}n hay diversas plataformas para ejecutarlas, como:

\begin{enumerate}
\item  Open Project

\item  IceScrum

\item  TeamWork Project

\item  X Planner

\item  Agile Mantis
\end{enumerate}
\end{enumerate}

\noindent 

\noindent 


\end{document}

